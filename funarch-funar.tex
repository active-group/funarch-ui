\documentclass[sigplan,screen]{acmart}

%%
%% \BibTeX command to typeset BibTeX logo in the docs
\AtBeginDocument{%
  \providecommand\BibTeX{{%
    Bib\TeX}}}

%% Rights management information.  This information is sent to you
%% when you complete the rights form.  These commands have SAMPLE
%% values in them; it is your responsibility as an author to replace
%% the commands and values with those provided to you when you
%% complete the rights form.
\setcopyright{acmcopyright}
\copyrightyear{2025}
\acmYear{2025}
\acmDOI{XXXXXXX.XXXXXXX}

%% These commands are for a PROCEEDINGS abstract or paper.
\acmConference[FUNARCH '25]{The Third ACM SIGPLAN Workshop on Functional Software Architecture -- FP in the Large}{October 17,
  2025}{Singapore}
%%
%%  Uncomment \acmBooktitle if the title of the proceedings is different
%%  from ``Proceedings of ...''!
%%
%%\acmBooktitle{Woodstock '18: ACM Symposium on Neural Gaze Detection,
%%  June 03--05, 2018, Woodstock, NY}
% \acmPrice{15.00}
% \acmISBN{978-1-4503-XXXX-X/18/06}


%%
%% Submission ID.
%% Use this when submitting an article to a sponsored event. You'll
%% receive a unique submission ID from the organizers
%% of the event, and this ID should be used as the parameter to this command.
%%\acmSubmissionID{123-A56-BU3}

%%
%% For managing citations, it is recommended to use bibliography
%% files in BibTeX format.
%%
%% You can then either use BibTeX with the ACM-Reference-Format style,
%% or BibLaTeX with the acmnumeric or acmauthoryear sytles, that include
%% support for advanced citation of software artefact from the
%% biblatex-software package, also separately available on CTAN.
%%
%% Look at the sample-*-biblatex.tex files for templates showcasing
%% the biblatex styles.
%%

%%
%% The majority of ACM publications use numbered citations and
%% references.  The command \citestyle{authoryear} switches to the
%% "author year" style.
%%
%% If you are preparing content for an event
%% sponsored by ACM SIGGRAPH, you must use the "author year" style of
%% citations and references.
%% Uncommenting
%% the next command will enable that style.
%%\citestyle{acmauthoryear}


%%
%% end of the preamble, start of the body of the document source.
\begin{document}

%%
%% The "title" command has an optional parameter,
%% allowing the author to define a "short title" to be used in page headers.
\title{Evolution of Functional UI Paradigms}

%%
%% The "author" command and its associated commands are used to define
%% the authors and their affiliations.
%% Of note is the shared affiliation of the first two authors, and the
%% "authornote" and "authornotemark" commands
%% used to denote shared contribution to the research.
\author{Markus Schlegel}
\email{markus.schlegel@active-group.de}
\affiliation{%
  \institution{Active Group GmbH}
  \streetaddress{Hechinger Straße 12/1}
  \city{Tübingen}
  \country{Germany}
  \postcode{72072}
}

\author{Michael Sperber}
\email{michael.sperber@active-group.de}
\affiliation{%
  \institution{Active Group GmbH}
  \streetaddress{Hechinger Straße 12/1}
  \city{Tübingen}
  \country{Germany}
  \postcode{72072}
}

%%
%% The abstract is a short summary of the work to be presented in the
%% article.
\begin{abstract}
  Functional paradigms for user-interface (UI) programming have
  undergone significant evolution over the years, from early
  monad-based approaches mimicking OO practice to modern
  model-view-update frameworks.  Changing from the inherently
  imperative classic Model-View-Controller pattern to functional
  approaches has significant architectural impact, drasticallyg
  reducing coupling and improving testability and maintainability.  On
  the other hand, achieving good modularity with functional approaches
  is an ongoing challenge.  This paper traces the evolution of
  functional UI toolkits along with the architectural implications of
  their designs, summarizes the current state of the art and discusses
  remaining issues.
\end{abstract}

%%
%% The code below is generated by the tool at http://dl.acm.org/ccs.cfm.
%% Please copy and paste the code instead of the example below.
%%
\begin{CCSXML}
<ccs2012>
   <concept>
       <concept_id>10010520.10010521</concept_id>
       <concept_desc>Computer systems organization~Architectures</concept_desc>
       <concept_significance>500</concept_significance>
       </concept>
   <concept>
       <concept_id>10011007.10010940.10010971.10010972</concept_id>
       <concept_desc>Software and its engineering~Software architectures</concept_desc>
       <concept_significance>500</concept_significance>
       </concept>
   <concept>
       <concept_id>10011007.10011006.10011008.10011009.10011012</concept_id>
       <concept_desc>Software and its engineering~Functional languages</concept_desc>
       <concept_significance>500</concept_significance>
       </concept>
 </ccs2012>
\end{CCSXML}

\ccsdesc[500]{Computer systems organization~Architectures}
\ccsdesc[500]{Software and its engineering~Software architectures}
\ccsdesc[500]{Software and its engineering~Functional languages}

%%
%% Keywords. The author(s) should pick words that accurately describe
%% the work being presented. Separate the keywords with commas.
\keywords{functional programming, software architecture, education}
%% A "teaser" image appears between the author and affiliation
%% information and the body of the document, and typically spans the
%% page.
\received{1 June 2023}

% FIXME revise

%%
%% This command processes the author and affiliation and title
%% information and builds the first part of the formatted document.
\maketitle

\section{UI and Software Architecture}

FIXME: basic terminology and thesis

\section{Model-View-Controller}

% http://bildungsgueter.de/Smalltalk/Pages/MVCTutorial/Pages/FS0004.htm

The most prominent early example for the design of UI programs and
toolkits is Trygve Reenskaug's \textit{Movel-View-Controller} pattern
(MVC)~\cite{MVC}.  The central innovation of MVC idea was the
decoupling of UI code (``view'') from the domain logic (``model''),
with the specific intent of allowing multiple different views on the
same model.

MVC is---at least nominally---the ancestor of most contemporary UI
paradigms and frameworks.  The separation of the model from the view
is universally beneficial from an architectural standpoint.
Developers can change the UI and replace the UI toolkit without
changing the model.  Moreover, since the model can function without
the UI, testing automation is significantly easier compared to a
tightly coupled view.  (If the only way to manipulate the model is
through the UI, automated tests need to remote-control the UI, which
is tedious to implement and brittle.)

MVC programs need to keep the view current when the state of the model
changes.  To that end, models typically implement a variation of the
\textit{observer pattern}~\cite{GoF}: The model maintains a list of
``dependents,'' i.e.\ objects that it notifies on changes by calling
an \texttt{update:} method.  (The controller is tightly coupled to the
view, and its role is unimportant for this discussion.)

As the \texttt{update:} model informs the view of state changes, this
approach is inherently imperative.  MVC has nice modularity
properties, as each part of the model only needs to be coupled to
``its'' part of the view.  It also creates several challenges to
software architects:
%
\begin{enumerate}
\item Keeping the view current with respect for the model involves two
  tasks: initially \emph{constructing} the view by creating the
  various UI widgets, and later \emph{mutating} the view upon calls to
  \texttt{update:}.  Conceptually, the result of mutating the view
  should be the same as re-constructing the view, but the concrete
  code for both is quite different, implementing the same logic twice.
\item To achieve modularity, architects should like to split a large,
  complex view into loosely coupled subviews.  This raises the
  question of how big those subviews should be. Making them large and
  correspondingly having them represent a largher chunk of the model
  simplifies the change logic, as it requires fewer implementations of
  \texttt{update:}, However it also increases coupling between model
  and view.  Moreover, it causes potential performance issues when the
  typical change is only to a small part of the model and only needs a
  small part of the view to change: In that case, \texttt{update:}
  either spends time drilling down to this small part of the view or
  changing larger parts of the view than necessary.
\item The view is not just a passive display, but will typically
  include interactive elements that cause changes in the model.
  Naively implemented, these changes indirectly trigger calls to
  \texttt{update:} in the view, which again might spill over into
  changes, causing a cyclic call chain.
\end{enumerate}
%
Consider the following practical example:
%
% weather composite, air pressure, temperature controls

\begin{figure}[tb]
\begin{verbatim}
class Weather(field pressure: Pressure,
              field temperature: Temperature) {
}

interface Observer {
  method update(info)
}

class Model {
  var observers: List Observer

  method changed(info) {
    for each sobserver in self.observers
      observer.update(info)

  method subscribe(observer) {
    observers += observer
  }
}

class Pressure extends Model {
  var hPa: double
  method setHPa(newHPa: double) {
    self.hPA = newHPa
    self.changed(newHPA)
  }
}

class Temperature extends Model {
  var kelvin: double
  method setKelvin(newKelvin: double) {
    self.kelvin = newKelvin
    self.changed(newKelvin)
  }
}
\end{verbatim}
  \caption{Model for weather data}
  \label{fig:weather-model}
\end{figure}

\begin{figure}[tb]
\begin{verbatim}
class TemperatureView
  (field temperatureModel: Temperature):
{
  var textField =
    new TextField(toText(temperatureModel.kelvin) + "K")

  temperatureModel.subscribe(self)

  method update(newKelvin) {
    textField.setText(toText(newKelvin) + "K")
  }
}

class PressureView
  (field pressureModel: Pressure)
{
  var textField =
    new TextField(toText(pressureModel.hPa) + "hPa")

  pressureModel.subscribe(self)

  method update(newHPa) {
    textField.setText(toText(newKelvin) + "hPa")
  }
}

class WeatherView(field weatherModel: Weather) {
  var temperatureView =
    new TemperatureView(weatherModel.temperature)
  var pressureView =
    new PressureView(weatherModel.pressure)
  ...
}
\end{verbatim}
  \caption{View for weather data}
  \label{fig:weather-view}
\end{figure}

Figure~\ref{fig:weather-model} shows object-oriented pseudcode for a
weather model that contains air pressure and temperature, each with
its own encapsulated state.  Figure~\ref{fig:weather-view} shows
skeleton code for a view that displays both in text fields.  The code
demonstrates nicely the modularity of the approach, as
\texttt{PressureView} and \texttt{TemperatureView} each subscribe to
their respective models, with no coordination required from
\texttt{WeatherView}.

The example also illustrates the first two challenges explained above:
%
\begin{enumerate}
\item The code for constructing the string representing the pressure
  and temperature is duplicated between the code that constructs the
  text field and the code that updates it.  It is possible to abstract
  out some code, but the fundamental separation between construction
  and update---which have to be consistent---remains.
\item The code contains two sub-views, updated separately.
  Alternatively, the subscription and update could happen between
  \texttt{WeatherView} and \texttt{Weather}, reducing modularity
  somewhat but also reducing the amount of code required.
\end{enumerate}
%
Note that the views are missing code to unsubscribe them from their
respective models---this further complicates the interaction between
model and view.

\begin{figure}[tb]
\begin{verbatim}
class TemperatureView2
  (field temperatureModel: Temperature) {
  var celsiusView =
    new TextField(toText(kToC(temperatureModel.kelvin)))
  var fahrenheitView =
    new TextField(toText(kToF(temperatureModel.kelvin)))

  temperatureModel.subscribe(new Observer {
      method update(newKelvin) {
        celsiusView.setText(toText(kToC(newKelvin)))
    }
  })
  temperatureModel.subscribe(new Observer {
      method update(newKelvin) {
        textField.setText(toText(kToC(newKelvin)))
    }
  })
  celsiusView.subscribe(new Observer {
     method update(...) {
      temperatureModel.setKelvin(cToK(...))
     }
  })
  fahrenheitView.subscribe(new Observer {
     method update(...) {
      temperatureModel.setKelvin(fToK(...))
     }
  })
}
\end{verbatim}
  %
  \caption{Two different sub-views on the same model}
  \label{fig:temperature-view2}
\end{figure}
%
Figure~\ref{fig:temperature-view2} shows a different view, just on the
temperature, but with two sub-views showing the temperature with
different units.  The code also adds the element of interactivity:
When the user interacts with the view, the two observers attached to
the views update the model.  This demonstrates the third challenge
described above: There is now an update loop between the model and
the two views: Each update to the model updates the view, which
updates the model, and so on.  The code could break the loop by
checking whether the new value is different from the old before
calling \texttt{changed}, but this is brittle---especially as
floating-point rounding is involved in the conversion between the
three units.

\section{The Curse of the Event Loop}


\section{Functional UIs: The Early Days}

Haggis, whatever Clean had

eXene~\cite{eXene} was an early specifically functional approach to UI
toolkits.  eXene built on Concurrent ML~\cite{ConcurrentML} (CML), and
replaced the event-loop/callback approach to reactivity by CML's
communicating primitives.  Specifically, a UI component would provide
a CML event that would emit information about user actions.  This
inverts the control flow relative to the event-loop model, as the
UI program can query the UI events for occurrences at its own leisure,
and combine them flexibly through CML's event combinators.

\section{The World Teachpack and React} 

\section{The evolution of Elm}

\section{UI Toolkits Elsewhere}

\section{The Modularity Challenge}

\subsection{From Reacl to Reacl-c}


\section{Persistent Pains}

%%
%% The next two lines define the bibliography style to be used, and
%% the bibliography file.
\bibliographystyle{ACM-Reference-Format}
\bibliography{funarch-ui}

\end{document}
\endinput
%%
% TODOs:
% - cite Conal

