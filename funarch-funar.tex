\documentclass[sigplan,screen]{acmart}

%%
%% \BibTeX command to typeset BibTeX logo in the docs
\AtBeginDocument{%
  \providecommand\BibTeX{{%
    Bib\TeX}}}

%% Rights management information.  This information is sent to you
%% when you complete the rights form.  These commands have SAMPLE
%% values in them; it is your responsibility as an author to replace
%% the commands and values with those provided to you when you
%% complete the rights form.
\setcopyright{acmcopyright}
\copyrightyear{2025}
\acmYear{2025}
\acmDOI{XXXXXXX.XXXXXXX}

%% These commands are for a PROCEEDINGS abstract or paper.
\acmConference[FUNARCH '25]{The Third ACM SIGPLAN Workshop on Functional Software Architecture -- FP in the Large}{October 17,
  2025}{Singapore}
%%
%%  Uncomment \acmBooktitle if the title of the proceedings is different
%%  from ``Proceedings of ...''!
%%
%%\acmBooktitle{Woodstock '18: ACM Symposium on Neural Gaze Detection,
%%  June 03--05, 2018, Woodstock, NY}
% \acmPrice{15.00}
% \acmISBN{978-1-4503-XXXX-X/18/06}


%%
%% Submission ID.
%% Use this when submitting an article to a sponsored event. You'll
%% receive a unique submission ID from the organizers
%% of the event, and this ID should be used as the parameter to this command.
%%\acmSubmissionID{123-A56-BU3}

%%
%% For managing citations, it is recommended to use bibliography
%% files in BibTeX format.
%%
%% You can then either use BibTeX with the ACM-Reference-Format style,
%% or BibLaTeX with the acmnumeric or acmauthoryear sytles, that include
%% support for advanced citation of software artefact from the
%% biblatex-software package, also separately available on CTAN.
%%
%% Look at the sample-*-biblatex.tex files for templates showcasing
%% the biblatex styles.
%%

%%
%% The majority of ACM publications use numbered citations and
%% references.  The command \citestyle{authoryear} switches to the
%% "author year" style.
%%
%% If you are preparing content for an event
%% sponsored by ACM SIGGRAPH, you must use the "author year" style of
%% citations and references.
%% Uncommenting
%% the next command will enable that style.
%%\citestyle{acmauthoryear}


%%
%% end of the preamble, start of the body of the document source.
\begin{document}

%%
%% The "title" command has an optional parameter,
%% allowing the author to define a "short title" to be used in page headers.
\title{Evolution of Functional UI Paradigms}

%%
%% The "author" command and its associated commands are used to define
%% the authors and their affiliations.
%% Of note is the shared affiliation of the first two authors, and the
%% "authornote" and "authornotemark" commands
%% used to denote shared contribution to the research.
\author{Markus Schlegel}
\email{markus.schlegel@active-group.de}
\affiliation{%
  \institution{Active Group GmbH}
  \streetaddress{Hechinger Straße 12/1}
  \city{Tübingen}
  \country{Germany}
  \postcode{72072}
}

\author{Michael Sperber}
\email{michael.sperber@active-group.de}
\affiliation{%
  \institution{Active Group GmbH}
  \streetaddress{Hechinger Straße 12/1}
  \city{Tübingen}
  \country{Germany}
  \postcode{72072}
}

%%
%% The abstract is a short summary of the work to be presented in the
%% article.
\begin{abstract}
  Functional paradigms for user-interface (UI) programming have
  undergone significant evolution over the years, from early
  monad-based approaches mimicking OO practice to modern
  model-view-update frameworks.  Changing from the inherently
  imperative classic Model-View-Controller pattern to functional
  approaches has significant architectural impact, drastically
  reducing coupling and improving testability and maintainability.  On
  the other hand, achieving good modularity with functional approaches
  is an ongoing challenge.  This paper traces the evolution of
  functional UI toolkits along with the architectural implications of
  their designs, summarizes the current state of the art and discusses
  remaining issues.
\end{abstract}

%%
%% The code below is generated by the tool at http://dl.acm.org/ccs.cfm.
%% Please copy and paste the code instead of the example below.
%%
\begin{CCSXML}
<ccs2012>
   <concept>
       <concept_id>10010520.10010521</concept_id>
       <concept_desc>Computer systems organization~Architectures</concept_desc>
       <concept_significance>500</concept_significance>
       </concept>
   <concept>
       <concept_id>10011007.10010940.10010971.10010972</concept_id>
       <concept_desc>Software and its engineering~Software architectures</concept_desc>
       <concept_significance>500</concept_significance>
       </concept>
   <concept>
       <concept_id>10011007.10011006.10011008.10011009.10011012</concept_id>
       <concept_desc>Software and its engineering~Functional languages</concept_desc>
       <concept_significance>500</concept_significance>
       </concept>
 </ccs2012>
\end{CCSXML}

\ccsdesc[500]{Computer systems organization~Architectures}
\ccsdesc[500]{Software and its engineering~Software architectures}
\ccsdesc[500]{Software and its engineering~Functional languages}

%%
%% Keywords. The author(s) should pick words that accurately describe
%% the work being presented. Separate the keywords with commas.
\keywords{functional programming, software architecture, education}
%% A "teaser" image appears between the author and affiliation
%% information and the body of the document, and typically spans the
%% page.
\received{1 June 2023}

FIXME revise

%%
%% This command processes the author and affiliation and title
%% information and builds the first part of the formatted document.
\maketitle

\section{Model-View-Controller}

\section{Functional UIs: The Early Days}

Haggis, whatever Clean had

\section{The World Teachpack and React} 

\section{The evolution of Elm}

\section{UI Toolkits Elsewhere}

\section{The Modularity Challenge}

\subsection{From Reacl to Reacl-c}


\section{Persistent Pains}

%%
%% The next two lines define the bibliography style to be used, and
%% the bibliography file.
\bibliographystyle{ACM-Reference-Format}
\bibliography{funarch-ui}

\end{document}
\endinput
%%
%% End of file `sample-sigplan.tex'.
